\usepackage[mathscr]{eucal}   % allows script versions of math fonts
\usepackage{extarrows}   %  allows arrows with stuff over and under them
\usepackage{bm}		% allows bolding of greek symbols
\usepackage{rotating}     % allows text and math to be rotated
\usepackage{graphicx}	% allows externally created pdfs to be included
\usepackage{booktabs}      % makes better tables
\usepackage{amsthm}      % allows style of theorem environments to change
\usepackage{amsmath}
\usepackage{natbib}
\usepackage{caption}
\usepackage{subfig}
\usepackage{algorithm}
\usepackage{algorithmic}
\usepackage{amssymb}
\usepackage{xypic}
\usepackage{setspace}
\usepackage{array}
\usepackage{mdwlist}
\usepackage{keystroke}
\usepackage{lineno}
\usepackage{upgreek}
\usepackage{appendix}
\usepackage{float}
\usepackage{endfloat}
%\usepackage[none]{hyphenat}
%\usepackage{indentfirst}



\input xy 
\xyoption{all}
\xyoption{graph}

%\usepackage{amssymb,amsmath,gensymb}
\usepackage{url}
\newcommand{\comment}[1]{}

\newcommand{\thj}{\ensuremath{\hat{\theta}_{(j)}}}       % define formatted theta
\newcommand{\thij}{\ensuremath{\hat{\theta}_{(i)(j)}}}   % define formatted theta
\newcommand{\thcv}{\bm{\hat{\theta}}_{ij}^{\mathrm{CV}}}
\newcommand{\thfv}{\bm{\hat{\theta}}_{j}^{\mathrm{FVO}}}
\newcommand{\M}{\ensuremath{\bm{\mathrm{M}}}}      % define the M matrix
\newcommand{\Mp}{\ensuremath{\bm{\mathrm{M'}}}}      % define the M' matrix
\newcommand{\Mh}{\ensuremath{\bm{\mathrm{\hat{M}}}}}      % define the M hat matrix
\newcommand{\y}{\ensuremath{\bm{\mathrm{y}}}}      % define the M matrix
\newcommand{\x}{\ensuremath{\bm{\mathrm{x}}}}      % define the M matrix
\newcommand{\z}{\ensuremath{\bm{\mathrm{z}}}}      % define the M matrix
\newcommand{\yh}{\ensuremath{\bm{\mathrm{\hat{y}}}}}      % define the M hat matrix
\newcommand{\link}{\ensuremath{\mathrm{link}}}
\newcommand{\mat}[1]{\bm{\mathrm{#1}}}


\bibpunct{(}{)}{;}{a}{}{,}  % gives different style for inline citations

%\renewcommand{\citep}[1]{(\citeauthor{#1} \citeyear{#1})}
\newcommand{\citeeg}[1]{(e.g. \citeauthor{#1} \citeyear{#1})}
\newcommand{\citenp}[1]{\citeauthor{#1} \citeyear{#1}}
\newcommand{\cites}[1]{\citeauthor{#1}'s (\citeyear{#1})}

\DeclareMathOperator*{\argmax}{arg\, max}
\DeclareMathOperator{\aic}{AIC}    % define AIC command
\DeclareMathOperator{\dev}{DEV}  % define DEV command

\theoremstyle{definition}			% pre-defined choices: plain, definition or remark
\newtheorem{principle}{Principle}	% define principle sections
\newtheorem{example}{Illustration}
\newtheorem{definition}{Definition}

% Here it is: the code that adjusts justification and spacing around caption.
\makeatletter
% http://www.texnik.de/floats/caption.phtml
% This does spacing around caption.
\setlength{\abovecaptionskip}{6pt}   % 0.5cm as an example
\setlength{\belowcaptionskip}{6pt}   % 0.5cm as an example
% This does justification (left) of caption.
\long\def\@makecaption#1#2{%
  \vskip\abovecaptionskip
  \sbox\@tempboxa{#1: #2}%
  \ifdim \wd\@tempboxa >\hsize
    #1: #2\par
  \else
    \global \@minipagefalse
    \hb@xt@\hsize{\box\@tempboxa\hfil}%
  \fi
  \vskip\belowcaptionskip}
\makeatother

