 \documentclass[12pt]{ecologyFORAPPENDIX}

\usepackage{tikz}
\usepackage{textpos}
\usepackage{verbatim}

\usepackage[mathscr]{eucal}   % allows script versions of math fonts
\usepackage{extarrows}   %  allows arrows with stuff over and under them
\usepackage{bm}		% allows bolding of greek symbols
\usepackage{rotating}     % allows text and math to be rotated
\usepackage{graphicx}	% allows externally created pdfs to be included
\usepackage{booktabs}      % makes better tables
\usepackage{amsthm}      % allows style of theorem environments to change
\usepackage{amsmath}
\usepackage{natbib}
\usepackage{caption}
\usepackage{subfig}
\usepackage{algorithm}
\usepackage{algorithmic}
\usepackage{amssymb}
\usepackage{xypic}
\usepackage{setspace}
\usepackage{array}
\usepackage{mdwlist}
\usepackage{keystroke}
\usepackage{lineno}
\usepackage{upgreek}
\usepackage{appendix}
\usepackage{float}
\usepackage{endfloat}
%\usepackage[none]{hyphenat}
%\usepackage{indentfirst}



\input xy 
\xyoption{all}
\xyoption{graph}

%\usepackage{amssymb,amsmath,gensymb}
\usepackage{url}
\newcommand{\comment}[1]{}

\newcommand{\thj}{\ensuremath{\hat{\theta}_{(j)}}}       % define formatted theta
\newcommand{\thij}{\ensuremath{\hat{\theta}_{(i)(j)}}}   % define formatted theta
\newcommand{\thcv}{\bm{\hat{\theta}}_{ij}^{\mathrm{CV}}}
\newcommand{\thfv}{\bm{\hat{\theta}}_{j}^{\mathrm{FVO}}}
\newcommand{\M}{\ensuremath{\bm{\mathrm{M}}}}      % define the M matrix
\newcommand{\Mp}{\ensuremath{\bm{\mathrm{M'}}}}      % define the M' matrix
\newcommand{\Mh}{\ensuremath{\bm{\mathrm{\hat{M}}}}}      % define the M hat matrix
\newcommand{\y}{\ensuremath{\bm{\mathrm{y}}}}      % define the M matrix
\newcommand{\x}{\ensuremath{\bm{\mathrm{x}}}}      % define the M matrix
\newcommand{\z}{\ensuremath{\bm{\mathrm{z}}}}      % define the M matrix
\newcommand{\yh}{\ensuremath{\bm{\mathrm{\hat{y}}}}}      % define the M hat matrix
\newcommand{\link}{\ensuremath{\mathrm{link}}}
\newcommand{\mat}[1]{\bm{\mathrm{#1}}}


\bibpunct{(}{)}{;}{a}{}{,}  % gives different style for inline citations

%\renewcommand{\citep}[1]{(\citeauthor{#1} \citeyear{#1})}
\newcommand{\citeeg}[1]{(e.g. \citeauthor{#1} \citeyear{#1})}
\newcommand{\citenp}[1]{\citeauthor{#1} \citeyear{#1}}
\newcommand{\cites}[1]{\citeauthor{#1}'s (\citeyear{#1})}

\DeclareMathOperator*{\argmax}{arg\, max}
\DeclareMathOperator{\aic}{AIC}    % define AIC command
\DeclareMathOperator{\dev}{DEV}  % define DEV command

\theoremstyle{definition}			% pre-defined choices: plain, definition or remark
\newtheorem{principle}{Principle}	% define principle sections
\newtheorem{example}{Illustration}
\newtheorem{definition}{Definition}

% Here it is: the code that adjusts justification and spacing around caption.
\makeatletter
% http://www.texnik.de/floats/caption.phtml
% This does spacing around caption.
\setlength{\abovecaptionskip}{6pt}   % 0.5cm as an example
\setlength{\belowcaptionskip}{6pt}   % 0.5cm as an example
% This does justification (left) of caption.
\long\def\@makecaption#1#2{%
  \vskip\abovecaptionskip
  \sbox\@tempboxa{#1: #2}%
  \ifdim \wd\@tempboxa >\hsize
    #1: #2\par
  \else
    \global \@minipagefalse
    \hb@xt@\hsize{\box\@tempboxa\hfil}%
  \fi
  \vskip\belowcaptionskip}
\makeatother


%\renewcommand{\listfigurename}{Figure legends}  % changes the name of the list of figures
\linenumbers 

\title{Appendix:  Statistical details}
\runninghead{Statistical details}
\author{Steven C. Walker}
\coauthor{Guillaume Gu\'{e}nard, Beatrix Beisner, Pierre Legendre}
\runningauthor{Walker et al.}
\address{Universit\'{e} de Montr\'{e}al}
\email{steve.walker@utoronto.ca}

\renewcommand{\efloatseparator}{}
\newcommand{\processfloatnow}{
	\begingroup
	\let\cleardoublepage\relax
	\let\clearpage\relax
	\processdelayedfloats
	\endgroup
}


\begin{document}
\maketitle
% left justification with indented paragraphs
\raggedright
\parindent=1.5em

%%% PUT THIS SECTION IN A VIGNETTE ON THE WORKFLOW
%\section{Data organisation}

%One of the most challenging aspects of simultaneously analysing data on community composition, environmental variables, and traits is that all of the data do not easily fit into a single data table with rows representing replication and columns representing variables.  

\section{Data structures}

We organise the full data into an \texttt{R} data frame with 120 rows called \texttt{croche},
\vspace{-18pt}
\singlespace
\begin{verbatim}
        den day size pp        taxon
0.043572985 178 0.18  Y armoured rot
0.046999720 188 0.18  Y armoured rot
0.055094990 205 0.18  Y armoured rot
0.034424544 219 0.18  Y armoured rot
0.046606746 233 0.18  Y armoured rot
0.036939579 247 0.18  Y armoured rot
0.052501360 261 0.18  Y armoured rot
0.061854147 275 0.18  Y armoured rot
0.082367403 289 0.18  Y armoured rot
0.086218550 303 0.18  Y armoured rot
0.002758679 178 0.33  N      Bosmina
0.001100328 188 0.33  N      Bosmina
\end{verbatim}
\vspace{-13pt}
\hspace{35pt}
\vdots
\hspace{38pt}
\vdots
\hspace{20pt}
\vdots
\hspace{18pt}
\vdots
\hspace{40pt}
\vdots
\vspace{-10pt}
\begin{verbatim}
0.005737424 289 0.18  N unprotected rot
0.007387916 303 0.18  N unprotected rot
\end{verbatim}
\doublespace

\noindent The five columns are density, Julian day, body size, predator protection, and taxon.  Community ecology data with taxon-level traits often have this form, in which observations are repeated for some variables.  For example, \texttt{size}, \texttt{pp}, and \texttt{taxon} are not replicated in time, and therefore are repeated across rows associated with different \texttt{day}s.  However, \texttt{den} is replicated both taxonomically and in time, and so is not repeated.  We used the \texttt{multitable} package in \texttt{R}, which was designed to manipulate data of this form.  The summarised version of the data, \texttt{croche.sum}, is much reduced,

\vspace{-18pt}
\singlespace
\begin{verbatim}
 day        cwm
 178 -0.6990721
 188 -0.7211258
 205 -0.8397823
 219 -0.7441729
 233 -0.8008270
 247 -0.7302490
 261 -0.8693237
 275 -0.8711310
 289 -0.9327495
 303 -0.9144807
\end{verbatim}
\doublespace

\section{Justifying models used in the main text}

\subsection{Summarised-data-model justifications}

Because the relationship between \texttt{day} and \texttt{cwm} is roughly linear, we modelled it using,
\vspace{-18pt}
\singlespace
\begin{verbatim}
  croche.sum.lm <- lm(cwm ~ day, croche.sum)
\end{verbatim}
\doublespace

\subsection{Full-data-model justifications}

Our approach to choosing full-data-models used visual assessment of residuals to identify features of the data that should be explicitly modelled.  We use raw residuals, $y_{ij} - \hat{y}_{ij}$, for fixed effects models and normalised residuals for mixed-effects models, which have the benefit of being approximately normally distributed with mean zero and standard deviation one for each taxon if model assumptions are met \citep{PinheiroAndBates2000}.  For all model fits, \texttt{den} is square-root transformed and \texttt{size} and \texttt{day} are z-scored (subtract the mean and divide by the standard deviation).

We begin by fitting six simple fixed effects models using the \texttt{R} \texttt{lm} function,
\vspace{-18pt}
\singlespace
\begin{verbatim}
  lm(den ~ day, croche)
  lm(den ~ size, croche)
  lm(den ~ size + I(size^2), croche)
  lm(den ~ day * (size + I(size^2)), croche)
  lm(den ~ taxon, croche)
  lm(den ~ -1 + taxon + (day + I(day^2)):taxon, croche)
\end{verbatim}
\doublespace
Julian \texttt{day} alone explains very little variation in \texttt{den} (Figure \ref{fig:fixedeffectsresidualsBYtaxa}, \texttt{den} $\sim$ \texttt{day}).  In contrast with \texttt{day}, body \texttt{size} alone explains much more variation but there is a strong quadratic trend in the residuals (Figure \ref{fig:fixedeffectsresidualsBYtaxa}, \texttt{den} $\sim$ \texttt{size}), suggesting that moderately sized species are least abundant (Figure \ref{fig:fixedeffectsresidualsBYtaxa}, \texttt{den} $\sim$ \texttt{size} + \texttt{size} $\hat{}$ 2).

\begin{figure}
\includegraphics{fixedeffectsresidualsBYtaxa.pdf}
\caption{}
\label{fig:fixedeffectsresidualsBYtaxa}
\end{figure}
\processfloatnow


Although \texttt{day} explained very little variation on its own, it would be surprising if it was not an important explanatory variable at all; many environmental variables change throughout the season, and many of these variables are likely to have an effect on community composition.   
Perhaps \texttt{day} interacts with \texttt{size} to explain a more substantial portion of the variation in \texttt{den}?  However, the interaction between \texttt{day} and \texttt{size} was only able to explain a little more variation than the quadratic \texttt{size} model (Figure \ref{fig:fixedeffectsresidualsBYtaxa}, compare \texttt{den} $\sim$ \texttt{size} + \texttt{size} $\hat{}$ \texttt{2}) with \texttt{den} $\sim$ \texttt{day} * (\texttt{size} + \texttt{size} $\hat{}$ \texttt{2})).  Nevertheless, we will formally model this weak interaction in at least one model because such trait-environment interactions potentially provide insight into why taxa vary in their responses to gradients---a key question in trait-based ecology.

These first four fixed-effects models all show patterns that are diagnostic of taxon effects.  The most evident such pattern is the fact that armoured rotifers tend to have positive residuals (open circles in Figure \ref{fig:fixedeffectsresidualsBYtaxa}; top-middle panel in Figure \ref{fig:fixedeffectsresidualsBYmodel}).  Such taxon effects suggest that a model with \texttt{taxon} as a factor will explain a significant portion of variation.  Indeed, a fixed effects model with only \texttt{taxon} as a factor explains much more variation than the other four models (Figure \ref{fig:fixedeffectsresidualsBYtaxa}, \texttt{abnd} $\sim$ \texttt{taxon}).  However, there are still taxon-effects present in the residuals of this model; all taxa do not appear to all have the same residual variance (e.g. the residuals for the three rotifer taxa appear to be more variable than for the other taxa).  Finally, taxon-specific relationships between \texttt{den} and \texttt{day} explain some more variation, but do not completely eliminate the need for taxon-specific residual variances (Figure \ref{fig:fixedeffectsresidualsBYtaxa}, \texttt{abnd} $\sim$ \texttt{taxon * (day + day} $\hat{}$ \texttt{2})).

\begin{figure}
\includegraphics{fixedeffectsresidualsBYmodel.pdf}
\caption{}
\label{fig:fixedeffectsresidualsBYmodel}
\end{figure}
\processfloatnow

In summation, Figures ??,?? suggest several effects that should be considered in our formal models: (1) a quadratic effect of \texttt{size}; (2) a \texttt{taxon}-specific effect of \texttt{day} (and possibly its square); (3) possibly a (weak) interaction between \texttt{day} and \texttt{size}; and (4) taxon-specific residual variances.  We fit a linear mixed effects model that includes effects 1, 2, and 4 using the \texttt{lme} function in the \texttt{R nlme} package,
\vspace{-12pt}
\singlespace 
\begin{verbatim}
  croche.lme <- lme(den ~ -1 + size + I(size 2),
    data = as.data.frame(croche),
    random = ~ day + I(day 2) | taxon,
    weights = varIdent(form = ~ 1 | taxon),
    method = "ML")
\end{verbatim}
\doublespace
Given this \texttt{croche.lme} object, we can efficiently fit another model that also includes all four effects using the \texttt{update} function to specify effect 3,
\begin{verbatim}
  croche.lme2 <- update(croche.lme, fixed. = . ~ . + size:day)
\end{verbatim}
Note that we have set \texttt{method = "ML"}, specifying maximum likelihood estimation.  We use maximum likelihood fitted models for likelihood ratio tests, which we describe below, but refit the models using restricted maximum likelihood (i.e. \texttt{method = "REML"}) for all other purposes because it is less biased \citep{PinheiroAndBates2000}.  Because our fixed-effect specifications are different between models, likelihood ratio tests based on restricted maximum likelihood estimates are not valid \citep{PinheiroAndBates2000}.

%To decide which of the first three effects should be fixed and which should be random, we treat the one involving \texttt{taxon} (i.e. effect 2) as random and all others as fixed (i.e. effects 1 and 3).  The for this decision is that in trait-based ecology, we are primarily interested in effects involving functional trait information (e.g. \texttt{size}) rather than those involving taxonomic information (e.g. \texttt{taxon}).  Therefore, taxon-effects are not the sources of variation that are of primary interest.  We treat taxon-effects as random 






%Taken together, these six fixed-effects models suggest that a better model for these data would include 


%We fit a mixed-effects model with these properties to the data.  The normalised residuals for this model show no obvious violations of its assumptions (Figure \ref{fig:randomeffectsresiduals}), conferring confidence in inferences draw from it.  For example, we infer that there is little evidence of an interaction between \texttt{day} and \texttt{size} \emph{per se}, with more support for interactions between \texttt{day} and other unmeasured (or unconsidered) traits as evidenced by the need to account for taxon-specific \texttt{day} effects.  Furthermore, we failed to reject our selected model against an alternative with a fixed effect of the interaction between \texttt{day} and \texttt{size} (LR $= 2.18$, $p = 0.14$).

\begin{figure}
\includegraphics{randomeffectsresiduals.pdf}
\caption{}
\label{fig:randomeffectsresiduals}
\end{figure}
\processfloatnow

\section{Inferences from the three models}

\subsection{Statistical significance of the \texttt{day} gradient}

Each of the three models used in the main text (Eqs. ??) involve the \texttt{day} gradient to some extent.  If this were not the case, then some models would predict a flat relationship between \texttt{day} and \texttt{cwm}, instead of the negative relationship actually observed.  To test the significance of \texttt{day} in the summarised data model (Eq.??) we conducted the standard F-test for linear models,
\vspace{-16pt}
\singlespace 
\begin{verbatim}
  anova(croche.sum.lm)  
\end{verbatim}
\doublespace
which returns,
\vspace{-16pt}
\singlespace 
\begin{verbatim}
            Df   Sum Sq  Mean Sq F value  Pr(>F)   
  week       1 0.045697 0.045697  18.884 0.00246 **
  Residuals  8 0.019359 0.002420
\end{verbatim}
\doublespace

To test the significance of \texttt{day} in the full data models (Eqs. ??) we used likelihood ratio tests.  The null model for these tests was a model with identical structure to the models actually used except that all of the model elements that involve \texttt{day} were removed,
\singlespace 
\begin{verbatim}
  croche.lme0 <- lme(den ~ -1 + size + I(size 2),
    data = as.data.frame(croche),
    random = ~ 1 | taxon,
    weights = varIdent(form = ~ 1 | taxon),
    method = "ML")
\end{verbatim}
\doublespace
The likelihood ratio tests themselves were conducted using the \texttt{anova} command as well,
\vspace{-16pt}
\singlespace 
\begin{verbatim}
  anova(croche.lme0, croche.lme)
  anova(croche.lme0, croche.lme2)
\end{verbatim}
\doublespace
which returns,
\vspace{-16pt}
\singlespace 
\begin{verbatim}
               Model df       AIC       BIC   logLik   Test  L.Ratio p-value
croche.lme0        1 15 -561.9217 -520.1093 295.9608                        
croche.lme         2 20 -589.9284 -534.1786 314.9642 1 vs 2 38.00671  <.0001

               Model df       AIC       BIC   logLik   Test  L.Ratio p-value
croche.lme0        1 15 -561.9217 -520.1093 295.9608                        
croche.lme2        2 21 -589.9082 -531.3708 315.9541 1 vs 2 39.98647  <.0001
\end{verbatim}
\doublespace

\subsection{Confidence intervals}

To calculate confidence intervals for model parameters, we used 1.96 times the standard errors returned by the \texttt{summary} method for \texttt{lme} model objects,
\vspace{-16pt}
\singlespace 
\begin{verbatim}
summary(croche.lme)
Linear mixed-effects model fit by REML
 Data: croche 
        AIC       BIC   logLik
  -575.5983 -520.1846 307.7992

Random effects:
 Formula: ~day + I(day^2) | taxon
 Structure: General positive-definite, Log-Cholesky parametrization
                 StdDev      Corr         
(Intercept)      0.044143352 (Intr)   day
day              0.012156500 0.286        
I(day^2)         0.009068011 0.142  0.232 
Residual         0.018749557              

Variance function:
 Structure: Different standard deviations per stratum
 Formula: ~1 | taxon 
 Parameter estimates:
   armoured rot         Bosmina      Cal adults        Cal cope    colonial rot 
      1.0000000       0.2213323       0.4448553       0.3179021       1.8399211 
    Cycl adults       Cycl cope     Daphnia cat     Daphnia l&d      Holopedium 
      0.8448272       0.4979797       0.5889436       0.2323527       0.5678671 
        nauplii unprotected rot 
      0.9290073       1.1161388 
Fixed effects: den ~ -1 + size + I(size^2) 
                Value  Std.Error DF   t-value p-value
size      -0.04712006 0.01366072 10 -3.449310  0.0062
I(size^2)  0.05285596 0.01042480 10  5.070214  0.0005
 Correlation: 
            size
I(size^2) -0.313

Standardized Within-Group Residuals:
        Min          Q1         Med          Q3         Max 
-2.16261609 -0.53025751 -0.05447561  0.67505672  1.85595280 

Number of Observations: 120
Number of Groups: 12 


summary(croche.lme2)
Linear mixed-effects model fit by REML
 Data: croche
        AIC       BIC   logLik
  -566.0385 -508.0329 304.0193

Random effects:
 Formula: ~day + I(day^2) | taxon
 Structure: General positive-definite, Log-Cholesky parametrization
                 StdDev      Corr         
(Intercept)      0.044065463 (Intr)   day
day              0.011063178 0.292        
I(day^2)         0.009073553 0.133  0.218 
Residual         0.018713669              

Variance function:
 Structure: Different standard deviations per stratum
 Formula: ~1 | taxon 
 Parameter estimates:
   armoured rot         Bosmina      Cal adults        Cal cope    colonial rot 
      1.0000000       0.2218467       0.4440647       0.3183614       1.8255520 
    Cycl adults       Cycl cope     Daphnia cat     Daphnia l&d      Holopedium 
      0.8437985       0.4991434       0.5887427       0.2328328       0.5684159 
        nauplii unprotected rot 
      0.9211510       1.2321081 
Fixed effects: den ~ size + I(size^2) + size:day -1 
                   Value   Std.Error  DF   t-value p-value
size         -0.05145452 0.014061772  10 -3.659177  0.0044
I(size^2)     0.05256191 0.010413627  10  5.047416  0.0005
size:day     -0.00513419 0.003553148 108 -1.444969  0.1514
 Correlation: 
           size I(size^2)
I(size^2)          -0.301       
size:day  0.241     0.022

Standardized Within-Group Residuals:
        Min          Q1         Med          Q3         Max 
-2.17134860 -0.51791924 -0.06856532  0.67461091  1.92405232 

Number of Observations: 120
Number of Groups: 12 
\end{verbatim}
\doublespace
















\subsection{Predicting with and without taxon effects}

Mixed effects models are capable of making two different types of predictions---called conditional and marginal predictions---depending on whether the random effects are used.  Ecologically, the difference between these two types is that conditional predictions make use of taxon effects whereas marginal predictions do not.  Mathematically, the reason for the difference arises because random effects are considered random realisation of a population of effects and so we may choose to make predictions using the average of the random effects across taxa (i.e. the marginal approach) or our estimates of the effects for each individual taxon (i.e. the conditional approach).  For models fitted using \texttt{lme}, the \texttt{fitted} function can return both types of fitted values by adjusting the \texttt{level} argument.  See the help file for more details (\texttt{?fitted.lme}).

\subsection{Predicting community-weighted means with full-data-models}

Predicting the summarised data with a full data model is the most technically nuanced aspect of our methodology.  The full data models (Eq.??) predict a distribution for each $y_{ij}$, which in turn induce a distribution for $z_i$ because $\bar{z}_i$ depends on $y_{ij}$ $(j = 1,...,m)$ (Eq.??).  The fitted value, $\hat{\bar{z}}$, for $\bar{z}_i$ is the expected value, E, (i.e. average) of this distribution,
\begin{equation}
\hat{\bar{z}}_i = \mathrm{E}(\bar{z}_i) = \mathrm{E}\left(\frac{\sum_{j=1}^m y_{ij} z_j}{\sum_{j=1}^m y_{ij}} \right)
\end{equation}
Unfortunately, there is no closed-form expression for this expected value, which is why these predictions are technically challenging.  However, there are several efficient ways to approximate $\hat{\bar{z}}$.

We begin with the most intuitive approach.  The fitted value for each $y_{ij}$ is given by $\hat{y}_{ij}$ (Eq.??), which is the expected value of $y_{ij}$ under the fitted model.  Therefore, one simple approximation for $\hat{\bar{z}}_i$ is to substitute $\hat{y}_{ij}$ for $y_{ij}$ into the definition of the community-weighted mean (Eq.??),
\begin{equation}
\hat{\bar{z}}_i \approx \frac{\sum_{j=1}^m \hat{y}_{ij} z_j}{\sum_{j=1}^m \hat{y}_{ij}}
\end{equation}
However, this prediction will usually be biased because $\bar{z}_i$ is a non-linear function of $y_{ij}$ $(j = 1,...,m)$, and a non-linear function at the average value of its argument does not usually equal the mean  of the function.  Nevertheless, in practice we have found that this approximation is very accurate.

To improve approximation ??, we use ideas from error propagation theory (refs??).  In particular, we add a bias correction term to Eq.??,
\begin{equation}
\hat{\bar{z}}_i \approx \frac{\sum_{j=1}^m \hat{y}_{ij} z_j}{\sum_{j=1}^m \hat{y}_{ij}} + \frac{1}{2} \sum_{i, j} \hat{\sigma}_j^2 \bar{z}''_{ij}
\end{equation}
where $\hat{\sigma}_j^2$ is the estimated conditional variance of $y_{ij}$ around $\hat{y}_{ij}$ and $\bar{z}''_{ij}$ is the second partial derivative of $\bar{z}_i$ with respect to $y_{ij}$.  With \texttt{lme} models, these conditional variances can be obtained using the \texttt{getVarCov} function and setting the \texttt{type} argument to \texttt{conditional}.  The second partial derivatives can be computed using,
\begin{equation}
\bar{z}''_{ij} = -2 \frac{\bar{z}'_{ij}}{\sum_k y_{ik}}
\end{equation}
where $\bar{z}'_{ij}$ is the first partial derivative,
\begin{equation}
\bar{z}'_{ij} = \frac{\sum_k y_{ik} z_j - \sum_k y_{ik} z_k}{\left( \sum_k y_{ik} \right)^2}
\end{equation}

In this paper we used Eq.?? to compute fitted community-weighted means.  However, there was very little difference between Eq.?? and Eq.?? (Figure ??), suggesting that the simpler method may often be good enough.  If better approximations are required, it is always possible to simulate from the conditional model, calculate the resulting community-weighted means, and average these over a number of replicate simulations.

\begin{figure}
\includegraphics{cwmcorrection.pdf}
\caption{}
\label{fig:cwmcorrection}
\end{figure}
\processfloatnow

\section{Gape-limitation hypothesis}

We further explore the gape-limitation hypothesis discussed briefly in the second-last paragraph of the Discussion.  To visualise the information in the data that relates to the hypothesis, we reproduce Figure ?? in the main text with re-ordered panels (Figure ??).  In this figure, all unprotected taxa (marked `N') are in the top two rows and all protected taxa (marked `Y') are in the bottom row.  Within these two groups, taxa are ordered by body size (numbers in parentheses).  Consider the protected taxa first; clearly the smaller taxa (i.e. armoured and colonial rotifers) are more positively related to \texttt{day} than the larger ones (i.e. \emph{Holopedium} and the protected \emph{Daphnia} species).  This pattern implies a negative \texttt{size}-\texttt{day} interaction among protected taxa.  In contrast, this \texttt{size}-\texttt{day} interaction is much less pronounced among the unprotected taxa.

To formally model these ideas, we fit the following modification of the model in main-text Eq. ?? to the data,
\begin{equation}
\mathtt{den} \sim \mathtt{size} + \mathtt{pp}:\mathtt{size}:\mathtt{day} + (\mathtt{day} + \mathtt{day}^2 | \mathtt{taxon})
\label{eq:fdminteraction}
\end{equation}
where \texttt{pp} is a categorical predator protection variable with two levels, \texttt{Y} and \texttt{N}.  We fitted this model as,
\begin{verbatim} 
  croche.lme3 <- update(croche.lme, fixed. = . ~ . + pp:size:day) 
\end{verbatim} 
In this model, protected and unprotected taxa have different \texttt{size}-\texttt{day} interactions, which was a significant addition to the model as determined with a likelihood ratio test (LR = 5.7, p = 0.017).  The \texttt{size}-\texttt{day} interaction was only significant for protected taxa (Table ??).  Table ?? gives the results for the simpler model in the main text (Eq.??) that does not consider predator protection.

\begin{figure}
\includegraphics{randomeffectsfitwithpredatorprotection.pdf}
\caption{}
\label{fig:randomeffectsfitwithpredatorprotection}
\end{figure}
\processfloatnow

\singlespace
\begin{table}
\caption{ANOVA table}
\begin{tabular}{lrrrrrr}
\hline
                              & Value  & Std.Error & DF  & t-value & p-value \\
\hline
$\mathtt{size}$ 		      & -0.051& 0.014 	& 10   & -3.6 & 0.0048 \\
\texttt{size}  $\hat{}$ \texttt{2} 	      & 0.052 & 0.011  & 10   & 4.7  & 0.0007 \\
$\mathtt{size:pp.N:day}$& -0.004 & 0.002 & 107 & -1.6 & 0.1154 \\
$\mathtt{size:pp.Y:day}$ & -0.023 & 0.006 & 107 & -3.7 & 0.0003 \\
\hline
\end{tabular}
\end{table}
\processfloatnow
\doublespace


\singlespace
\begin{table}
\caption{Another ANOVA table}
\begin{tabular}{lrrrrrr}
\hline
                              & Value  & Std.Error & DF  & t-value & p-value \\
\hline
$\mathtt{size}$ 		      & -0.051& 0.014 	& 10   & -3.7 & 0.0044 \\
\texttt{size}  $\hat{}$ \texttt{2} 	      & 0.053 & 0.010  & 10   & 5.0  & 0.0005 \\
$\mathtt{size:day}$ & -0.005 & 0.004 & 108 & -1.4 & 0.1514 \\
\hline
\end{tabular}
\end{table}
\processfloatnow
\doublespace

\bibliographystyle{ecology}
% ***   Set the bibliography file.   ***
% ("thesis.bib" by default; change if needed)
\bibliography{/Users/stevenwalker/Documents/Bibliography/Bibliography}

\end{document}